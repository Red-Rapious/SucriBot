%% LyX 2.3.6.2 created this file.  For more info, see http://www.lyx.org/.
%% Do not edit unless you really know what you are doing.
\documentclass[french]{report}
\usepackage[utf8]{inputenc}
\usepackage{geometry}
\geometry{verbose,tmargin=2cm,bmargin=2cm,lmargin=2cm,rmargin=2cm,headheight=1cm,headsep=1cm,footskip=1cm}
\pagestyle{headings}
\setcounter{secnumdepth}{3}
\setcounter{tocdepth}{3}
\setlength{\parindent}{0bp}
\usepackage{latexsym}
\usepackage{fancybox}
\usepackage{calc}
\usepackage{amssymb}
\usepackage{graphicx}
\usepackage{esint}

\makeatletter

%%%%%%%%%%%%%%%%%%%%%%%%%%%%%% LyX specific LaTeX commands.
%% Because html converters don't know tabularnewline
\providecommand{\tabularnewline}{\\}
\newcommand{\lyxadded}[3]{#3}
\newcommand{\lyxdeleted}[3]{}

\AtBeginDocument{
  \def\labelitemi{\(\Box\)}
  \def\labelitemii{\(\bullet\)}
  \def\labelitemiii{\(\odot\)}
  \def\labelitemiv{\(\circ\)}
}

\makeatother

\usepackage{babel}
\makeatletter
\addto\extrasfrench{%
   \providecommand{\og}{\leavevmode\flqq~}%
   \providecommand{\fg}{\ifdim\lastskip>\z@\unskip\fi~\frqq}%
}

\makeatother
\begin{document}
\include{nc2021}

\noindent\ovalbox{\begin{minipage}[t]{1\columnwidth - 2\fboxsep - 0.8pt}%
%
\end{minipage}}

\ouverture{TD25}{Lundi 10 avril}

\nex {\large{}$\heartsuit$} Une suite récurrente $u$ vérifiant
$u_{n+2}=12.u_{n}-u_{n+1}$ pour tout $n$ reste de signe constant.
Montrez que c'est une suite géométrique, et calculez $u_{100}/u_{5}$.\\

\nex Est il possible de choisir $a$ réel pour que les suites récurrentes
``$u_{n+2}=a.u_{n+1}-u_{n}$ pour tout $n$'' soient toutes périodiques
de période $9$ ? \\
Est il possible de choisir $a$ et $b$ rationnels pour que les suites
récurrentes ``$u_{n+2}=a.u_{n+1}+b.u_{n}$ pour tout $n$'' soient
toutes périodiques de période $12$ ? \\

Est il possible de choisir $a$ et $b$ irrationnels pour que les
suites récurrentes ``$u_{n+2}=a.u_{n+1}+b.u_{n}$ pour tout $n$''
soient toutes périodiques de période $12$ ? \\
Est-il possible de choisir $a$, $b$, $c$ et $d$ réels pour que
dans le couple $(u,\ v)$ de suites vérifiant $\left\{ \begin{array}{cccc}
u_{n+1} & = & a.u_{n} & +b.v_{n}\\
v_{n+1} & = & c.u_{n} & +d.v_{n}
\end{array}\right|$ pour tout $n$, la suite $u$ soit périodique de période $6$. \\
Montrez que si $u$ est périodique, alors $v$ l'est aussi.\\

\nex Si \texttt{\textbf{L}} est une liste, on définit \\
\fbox{\begin{minipage}[b][1\totalheight][t]{0.6\columnwidth}%
\texttt{\textbf{\small{}def lambyaucaramel(L) :}}~\\
\texttt{\textbf{\small{}....n = len(L)}}~\\
\texttt{\textbf{\small{}....M = {[}0 for k in range(n){]} for i in
range(n){]}}}~\\
\texttt{\textbf{\small{}....for i in range(n-1) :}}~\\
\texttt{\textbf{\small{}........M{[}i{]}{[}i{]}, M{[}i{]}{[}i+1{]},
M{[}i+1{]}{[}i{]} = L{[}i{]}, -1, 1}}~\\
\texttt{\textbf{\small{}....M{[}n-1{]},{[}n-1{]} = L{[}-1{]}}}~\\
\texttt{\textbf{\small{}....return(M)}}%
\end{minipage}}%
\begin{minipage}[b][1\totalheight][t]{0.38\columnwidth}%
$\dsp\frac{1}{L[-1]+\dsp\frac{1}{L[-2]+\dsp\frac{1}{L[-3]+\dsp\frac{1}{\ldots+\dsp\frac{1}{L[0]}}}}}$%
\end{minipage}\\
Comparez \texttt{\textbf{det(L)/det(L{[}:-1{]})}} et la fraction continuée
écrite à droite \emph{\small{}(et appelée à tort ``}{\small{}fraction
continue}\emph{\small{}'' dans certains livres)}.\\

\nex $\heartsuit$ $u_{0}$ donné. $u_{n+1}=3.Arcsin(u_{n})/\pi$.
Discuter.\\

\nex Montrez que $x\associe\sqrt{x}$ est lipschitzienne sur $[1,\ +\infty[$
\emph{\small{}(sans utiliser un théorème sur la dérivée)}.\\

\nex $\heartsuit$ Montrez que l'application $x\associe x^{2}$ est
lipschitzienne sur tout segment \emph{\small{}(en revenant à la définition,
et non pas en majorant la dérivée)}.\\

\nex On rappelle que pour un emprunt d'une somme de $S$ euros au
taux $\tau$ \footnote{le taux $\tau$, voilà une blague !} d'une
durée de $n$ mois avec des mensualités d'un montant $\mu$, on a
les formules suivantes : \\
$\mu=\dsp\frac{\tau.S}{1-(1+\tau)^{-n}}$ et $n=\dsp-\frac{\ln\Big(1-\dsp\frac{\tau.S}{\mu}\Big)}{\ln(1+\tau)}$
\emph{\small{}(redémontrez les)}.\\
Donnez les limites de ces quantités quand $\tau$ tend vers $0$.
Est ce cohérent ?\\

\nex $u_{n+1}=u_{n}+\dsp\frac{1+u_{n}}{1+2.u_{n}}$. Vous savez ce
qu'il vous reste à faire.\\

\nex Vérifiez $\sqrt{n!}\ll n^{n/2}\ll n!\ll n^{n}$ quand $n$ tend
vers l'infini (utilisez le lemme classique).\\

\nex Classez les suites qui suivent pour les relation $\sim$ \emph{\small{}(équivalent)},
$\ll$ \emph{\small{}(négligeable)} et $O$ \emph{\small{}(quotient
borné)}.\\
\begin{tabular}{|c|c|c|c|c|}
\hline 
$(\ln(n))^{n}$ & $(\log(n))^{n}$ & $n^{\ln(n)}$ & $\log(n^{n})$ & $\log_{n}(n)$\tabularnewline
\hline 
\hline 
$\log_{\ln(n)}(n)$ & $n^{\log(n)}$ & $\ln(n^{n})$ & $n^{\ln(n)/n}$ & $(\log_{n}(n))^{n}$\tabularnewline
\hline 
\end{tabular}\\
\emph{\small{}$\ln$ est le logarithme népérien. $\log$ est le logarithme
de base $10$. $\ln_{a}$ est le logarithme de base $a$.}\\

\nex $u_{0}$ est entre $0$ et $1$, et $u_{n+1}=\dsp\frac{\sqrt{u_{n}}}{\sqrt{u_{n}}+\sqrt{1-u_{n}}}$.
Étudiez la suite.\\

\nex En quels points du graphe de l'application $x\associe\dsp\frac{x-1}{x+3}$
la tangente au graphe passe-t-elle par l'origine du repère ?\\

\nex $\heartsuit$ On définit $u_{0}<0$ et $u_{n+1}=e^{u_{n}}-1$.
Montrez que $(u_{n})$ est négative. Montrez que $(u_{n})$ est croissante.
Montrez que $(u_{n}(u)$ converge et calculez sa limite.\\
On définit $u_{0}>0$ et $u_{n+1}=e^{u_{n}}-1$. Montrez que $(u_{n})$
est positive. Montrez que $(u_{n})$ est croissante. Montrez que $(u_{n})$
ne peut pas converger. Concluez.\\

\nex Un réel $\alpha$ est dit adhérent à une partie $A$ de $\R$
si et seulement si il existe une suite $(a_{n})$ de points de $A$
qui converge vers $\alpha$. Montrez que tout point de $A$ est adhérent
à $A$. Qui sont les points adhérents à $\Z$ ?\\
Montrez que tout points adhérent à $A$ est adhérent à $A\cup B$.
Montrez que tout point adhérent à $A\cup B$ est adhérent à $A$ ou
adhérent à $B$ {\small{}(d'une suite $(c_{n})$}\emph{\small{}, ne
gardez soit que les points de $A$ ou soit que les points de $B$)}.\\
Montrez par un contre-exemple que l'on peut être adhérent à $A$ et
à $B$ sans être adhérent à $A\cap B$.\\

\nex On rappelle les définitions :\\
{\large{}}%
\begin{tabular*}{1\columnwidth}{@{\extracolsep{\fill}}|c|c|c|}
\hline 
{\large{}$\mathring{A}$} & {\large{}intérieur de $A$} & {\large{}$a\in\mathring{A}\Leftrightarrow(\exists\alpha>0,\ [a-\alpha,\ a+\alpha]\subset A)$}\tabularnewline
\hline 
\hline 
{\large{}$\overline{A}$} & {\large{}adhérence de $A$} & {\large{}$c\in\overline{A}\Leftrightarrow(\forall\alpha>0,\ [c-\alpha,\ c+\alpha]\cap A\neq\emptyset$}\tabularnewline
\hline 
\end{tabular*}\\

Complétez %
\begin{tabular*}{0.85\columnwidth}{@{\extracolsep{\fill}}|c|c|c|c|c|}
\hline 
$A$ & $]0,\ 1]$ & $\Q$ & $([0,\ 1]\cap\Q\cup([1,\ 2]\cap(\R-\Q))$ & $\Z\cup]0,\ 1[$\tabularnewline
\hline 
\hline 
$\mathring{A}$ & $]0,\ 1[$ & $\emptyset$ &  & $]0,\ 1[$\tabularnewline
\hline 
$\overline{A}$ & $[0,\ 1]$ & $\R=]-\infty,\ +\infty[$ &  & $\Z\cup]0,\ 1[$\tabularnewline
\hline 
$\mathring{\overline{A}}$ & $]0,\ 1[$ & $\R$ &  & $]0,\ 1[$\tabularnewline
\hline 
$\overline{\mathring{A}}$ & $[0,\ 1]$ & $\emptyset$ &  & $[0,\ 1]$\tabularnewline
\hline 
$\overline{A}-\mathring{A}$ & $\{0,\ 1\}$ & $\R$ &  & $\Z$\tabularnewline
\hline 
\end{tabular*}\\
Trouvez un ensemble pour lequel les six lignes sont toutes distinctes.\\
\fbox{\begin{minipage}[t]{0.98\columnwidth}%
Pourquoi est ce que je ne demande rien sur $\overline{\overline{A}}$
ni sur $\mathring{\mathring{A}}$ ?%
\end{minipage}}\\
Montrez que le complémentaire de l'intérieur de $A$ est l'adhérence
du complémentaire de $A$.\\
Trouvez une phrase avec complémentaire de l'adhérence.\\

Lesquelles sont vraies \\
{\small{}}%
\begin{tabular}{cccccc}
{\small{}l'adhérence} & {\small{}de la réunion de A est B} & {\small{} est la réunion} & {\small{}de l'adhérence de $A$ et de l'adhérence de $B$} &  & \tabularnewline
{\small{}l'adhérence} & {\small{}de l'intersection de A est B} & {\small{} est l'intersection} & {\small{}de l'adhérence de $A$ et de l'adhérence de $B$} &  & \tabularnewline
{\small{}l'intérieur} & {\small{}de la réunion de A est B} & {\small{} est la réunion} & {\small{}de l'intérieur de $A$ et de l'intérieur de $B$} &  & \tabularnewline
{\small{}l'intérieur} & {\small{}de l'intersection de A est B} & {\small{} est l'intersection} & {\small{}de l'intérieur de $A$ et de l'intérieur de $B$} &  & \tabularnewline
\end{tabular}{\small{}}\\

\nex On définit, pour $f$ continue de $[0,\ 1]$ dans $\R$ : \\
\begin{tabular}{|c|c|c|}
\hline 
$N_{1}(f)=\dsp\int_{0}^{1}|f(t)|.dt$ & $N_{2}(f)=\dsp\sqrt{\int_{0}^{1}(f(t))^{2}.dt}$ & $N_{\infty}(f)=Sup\{|f(t)|.dt|\ t\in[0,\ 1]\}$\tabularnewline
\hline 
\end{tabular}\\
Calculez ces trois normes pour les applications suivantes : $t\associe t^{n}$,
$\exp$, $t\associe\sin(\pi.t)$.\\
Montrez : $N_{1}\leqslant N_{\infty}$(c'est à dire $\forall f,\ N_{1}(f)\leqslant N_{\infty(f)}$),
$N_{2}\leqslant N_{\infty}$ et $N_{1}\leqslant N_{2}$ (là, il faut
penser à quelquechose, ou à quelqu'un ou à quelques uns).\\
L'application $f\associe|f(0)|+N_{1}(f')$ est elle une norme sur
$C_{1}([0,\ 1],\R)$ (les applications dérivables à dérivée continue).\\

\nex %
\begin{minipage}[b][1\totalheight][t]{0.48\columnwidth}%
Pour tout $n$, on pose $f_{n}=x\associe Arctan((n+1).x)-Arctan(n.x)$.
Montrez (non, sans en revenir aux $\varepsilon$) que chaque $f_{n}$
est continue de même que chaque $\dsp\sum_{n=0}^{N}f_{n}$ (notée
$F_{N}$). Montrez que $F_{\infty}$ (notation abusive) n'est pas
continue en $0$.\\
Sinon, que pensez vous de $x\associe x^{n}$ sur $[0,\ 1]$ ?\\
Ou $x\associe\sin(x))^{2.n}$ sur $\R$ tout entier ?%
\end{minipage}%
\begin{minipage}[b][1\totalheight][t]{0.5\columnwidth}%
\includegraphics[width=1\columnwidth,bb = 0 0 200 100, draft, type=eps]{imageExercices22_1.png}%
\end{minipage}\\

\nex Une entreprise fabrique et vend des bloukfons. Le coût de fabrication
est de $20$ euros pièce. Elle les vend $40$ euros. Chaque mois,
elle en vend $10.000$. Calculez son bénéfice net. Elle constate quand
même que chaque fois qu'elle baisse le prix de vente de dix centimes
d'euros, elle en vend $100$ de plus par mois. Quel sera le prix optimal
de vente pour maximiser son bénéfice ? Et c'est quoi un bloukfon ?\\

\nex{\large{} $\heartsuit$} Montrez que si $f$ est lipschitzienne
de $[0,\ 1]$ dans $\R^{+*}$ alors $1/f$ l'est aussi \emph{\small{}(il
faudra penser à introduire $Inf(f(x)\ |\ x\in[0,\ 1])$)}.\\

\nex $\heartsuit$ On définit $f=x\associe\exp([\ln(x)])$. Prolongez
la par continuité en $0$ (encadrement, pas d'epsilon). Est elle alors
dérivable ?\\
Calculez $\dsp\int_{0}^{1}f(t).dt$.\\

\nex On travaille avec les entiers de $0$ à $16$ pour l'addition
et la multiplication modulo $17$. \\
Complétez %
\begin{tabular}{|c|c|c|c|c|c|c|c|c|c|c|c|c|c|c|c|c|}
\hline 
$a$ & $1$ & $2$ & $3$ & $4$ & $5$ & $6$ & $7$ & $8$ & $9$ & $10$ & $11$ & $12$ & $13$ & $14$ & $15$ & $16$\tabularnewline
\hline 
\hline 
inverse de $a$ & {*} & {*} & {*} & $13$ & {*} &  & $5$ &  &  & {*} &  & {*} & {*} & {*} & {*} & {*}\tabularnewline
\hline 
\end{tabular}\\
Complétez $\left(\begin{array}{ccc}
2 & 4 & *\\
1 & * & 3\\
* & 1 & *
\end{array}\right)$ pour que ce soit l'inverse de $\left(\begin{array}{ccc}
7 & 2 & 2\\
15 & 5 & 16\\
16 & 2 & 0
\end{array}\right)$.\\
Calculez le déterminant de $\left(\begin{array}{ccc}
7 & 2 & 2\\
15 & 5 & 16\\
16 & 2 & 0
\end{array}\right)$ en utilisant la formule $\dsp\sum_{\sigma\in S_{n}}Sgn(\sigma).a_{1}^{\sigma(1)}\ldots a_{n}^{\sigma(n)}$.\\
Résolvez l'équation $\left\{ \begin{array}{ccccc}
7.a & +2.b & +2.c & = & 2\\
15.a & +5.b & +16.c & = & 1\\
16.a & +2.b &  & = & 4
\end{array}\right|$.\\

\nex Montrez que si $f$ continue de $\R$ dans $\R$ vérifie $\dsp\int_{-1}^{1}f(t).dt=0$
alors elle admet au moins un point fixe \emph{\small{}(on pourra étudier
$f-Id$ sur $[-1,\ 1]$ et montrer qu'elle ne peut pas rester de signe
constant)}.\\

\nex %
\begin{minipage}[b][1\totalheight][t]{0.55\columnwidth}%
Placez aux huit sommets d'un cube les entiers de 1 à 8.\\
Il faut ensuite que pour chaque face, le produit des quatre entiers
aux quatre coins de la face soit la valeur imposée sur le patron du
développement du cube ci contre :%
\end{minipage}\texttt{\textbf{\LARGE{}}}%
\begin{minipage}[b][1\totalheight][t]{0.4\columnwidth}%
$.\quad$ \texttt{\textbf{\LARGE{}}}%
\begin{tabular}{c|c|cc}
\cline{2-2} 
 & \texttt{\textbf{\LARGE{}280}} &  & \tabularnewline
\hline 
\multicolumn{1}{|c|}{\texttt{\textbf{\LARGE{}96}}} & \texttt{\textbf{\LARGE{}1344}} & \multicolumn{1}{c|}{\texttt{\textbf{\LARGE{}420}}} & \multicolumn{1}{c|}{\texttt{\textbf{\LARGE{}30}}}\tabularnewline
\hline 
 & \texttt{\textbf{\LARGE{}144}} &  & \tabularnewline
\cline{2-2} 
\end{tabular}%
\end{minipage}\texttt{\textbf{\LARGE{}}}~\\

\nex Soit $f$ une application continue de $\R$ dans $\R$. On suppose
que $f$ n'a pas de point fixe et vérifie $f(0)>0$. Montrez : $\forall x\in\R,\ f(x)>x$.
Déduisez $\forall x\in\R,\ f(f(x))>x$. Déduisez le résultat suivant
: si $g\circ g$ admet au moins un point fixe, alors $g$ admet au
moins un point fixe.\\

\nex %
\begin{minipage}[b][1\totalheight][t]{0.48\columnwidth}%
Soit $f$ une application de classe $C^{1}$ de $[a,\ b]$ dans $\R$.
La longueur de l'arc de la courbe représentative de $f$ est donnée
par la formule $\dsp\int_a^b\sqrt{1+(f'(t))^2}.dt$ \emph{\small{}(que
l'on ne justifiera que plus tard dans l'année et qui était donnée
et admise aussi dans le sujet de concours }\footnote{\emph{\small{}C.C.P filière PSI 2014}}\emph{\small{}
dont je me suis ici inspiré)}. Vérifiez la formule pour $f=t\associe\alpha.t$
sur le segment $[a,\ b]$.\\
Vérifiez la formule pour $f=t\associe\sqrt{1-t^{2}}$ sur le segment
$[-1/2,\ 1/2]$ puis sur un segment $[a,\ b]$ inclus dans $[-1,\ 1]$.%
\end{minipage}%
\begin{minipage}[b][1\totalheight][t]{0.48\columnwidth}%
\includegraphics[width=1\columnwidth,bb = 0 0 200 100, draft, type=eps]{imageDS4_1.png}%
\end{minipage}\\
Calculez la longueur du graphe du cosinus hyperbolique sur $[0,\ 1]$.\\
Calculez la longueur du graphe de $t\associe t^{2}$ sur $[0,\ 1]$
\emph{\small{}(vous serez amené(e) à effectuer un changement de variable
hyperbolique)}.\\
\begin{minipage}[b][1\totalheight][t]{0.48\columnwidth}%
Pour tout $n$, on note $L_{n}$ la longueur de l'arc de $x\associe x^{n}$
entre $0$ et $1$. Donnez la forme intégrale de cette longueur. Calculez
$L_{1}$ et $L_{2}$.\\
Émettez une conjecture sur la limite de $L_{n}$ quand $n$ tend vers
l'infini.%
\end{minipage}%
\begin{minipage}[b][1\totalheight][t]{0.48\columnwidth}%
\includegraphics[width=1\columnwidth,bb = 0 0 200 100, draft, type=eps]{imageDS04_4.png}%
\end{minipage}\\
On pose pour tout $n$ $\varphi_{n}=t\associe\dsp\frac{1}{\sqrt{1+n^{2}.t^{2.n-2}}+n.t^{n-1}}$,
$\mu_{n}=\dsp\int_{0}^{1}\varphi_{n}(t).dt$. Montrez alors $L_{n}=n.\dsp\int_{0}^{1}t^{n-1}.dt+\mu_{n}$.\\
Montrez : $\mu_{n}\leqslant1$ et $L_{n}<2$ pour tout $n$. Pour
tout $n$, on pose $a_{n}=e^{-2.\frac{\ln(n)}{n-1}}$. \\
Montrez : $\forall t\in[0,\ a_{n}],\ \varphi_{n}(t)\geqslant\dsp\frac{1}{\sqrt{1+\dsp\frac{1}{n^{2}}}+\dsp\frac{1}{n}}$.
Déduisez $\mu_{n}\geqslant\dsp\frac{a_{n}}{\sqrt{1+\dsp\frac{1}{n^{2}}}+\dsp\frac{1}{n}}$.\\
Montrez par encadrement que la suite $(\mu_{n})$ admet pour limite
$1$ quand $n$ tend vers l'infini. Validez alors votre conjecture.\\
\trait %
\begin{minipage}[b][1\totalheight][t]{0.48\columnwidth}%
Montrez pour $f$ $C^{1}$ de $[0,\ 1]$ dans $\R$, croissante et
vérifiant $f(0)=0$ et $f(1)=1$, alors la longueur du graphe de $f$
est entre $1$ et $2$ \emph{(indication : $\sqrt{1+a^{2}}$ contre
$1+a$)}.%
\end{minipage}%
\begin{minipage}[b][1\totalheight][t]{0.48\columnwidth}%
\\
\includegraphics[width=1\columnwidth,bb = 0 0 200 100, draft, type=eps]{imageDS04_5.png}%
\end{minipage}\\
\trait %
\begin{minipage}[b][1\totalheight][t]{0.48\columnwidth}%
On prend l'application $t\associe1/t$ \emph{\small{}(arc d'hyperbole)}.
Le segment sera $[1/2,\ 1]$. Donnez l'expression intégrale de la
longueur \emph{\small{}(mais ne la calculez pas)}.\\
Montrez que c'est aussi la longueur du graphe sur $[1,\ 2]$ \emph{\small{}(justifiez
par changement de variable et aussi par un argument géométrique)}.%
\end{minipage}%
\begin{minipage}[b][1\totalheight][t]{0.48\columnwidth}%
\includegraphics[width=1\columnwidth,bb = 0 0 200 100, draft, type=eps]{imageDS04_3.png}%
\end{minipage}\\

\nex Le théorème de convergence logarithmique dit \og soit $(a_{n})$
une suite, on suppose que $\Big(\dsp\frac{a_{n+1}}{a_{n}}\Big)$ converge
vers une une limite $\lambda$ dans $[0,\ 1[$, on déduit que $(a_{n})$
converge vers $0$\fg . Un élève dont je tairai le nom\footnote{à vous d'eninventer un qui fasse un mauvais calembour}
a inventé une réciproque. Montrez qu'il a tort. Et dites moi qui c'est.\\
Mais si $(a_{n})$ est monotone ?\\

\nex 

Faux : si $u_{n}$ tend vers $a$ quand $n$ tend vers l'infini, alors
$u_{n+1}.u_{n+2}\ldots u_{2n}$ tend vers $a^{n}$ quand $n$ tend
vers $+\infty$.\\
Vrai ou faux : si $u_{n}$ tend vers $a>1$ quand $n$ tend vers l'infini,
alors $u_{n+1}.u_{n+2}\ldots u_{2n}$ est équivalent à $a^{n}$ quand
$n$ tend vers $+\infty$.\\

\nex Algorithme de Borchardt celui trigonométrique : on donne $0<a_{0}<b_{0}$
, puis pour tout $n$ de $\N$ : $a_{n+1}=\dsp\frac{a_{n}+b_{n}}{2}$
et $b_{n+1}=\sqrt{a_{n+1}.b_{n}}$. \\
(a) Montrer que $(a_{n})$ et $(b_{n})$ sont convergentes de même
limite, que l’on notera $\lambda$. \\
(b) On pose $q_{n}=\dsp\frac{a_{n}}{b_{n}}$ ; vérifiez : $q_{n+1}=\sqrt{\dsp\frac{1+q_{n}}{2}}$.\\
(c) Déduisez $a_{n}=b_{n}.\cos(\alpha/2^{n})$ où $\alpha$ est une
mesure angulaire que vous définirez , puis $b_{n}=\dsp\frac{b_{0}.\sin(\alpha)}{2^{n}.\sin(\alpha/2^{n})}$
. \\
(d) Déduisez : $\lambda=\dsp\frac{b_{0}.\sin(\alpha)}{\alpha}=\frac{\sqrt{b_{0}^{2}-a_{0}^{2}}}{Arccos(b_{0}/a_{0})}$
et $(b_{n}-a_{n})\sim_{n\rightarrow+\infty}b_{0}.\alpha.\sin(\alpha).2^{-2n-1}$
. \\
(e) Que vaut donc $\lambda$ dans le cas $a_{0}=1/2$ et $b_{0}=1/\sqrt{2}$
?\\

\nex $\clubsuit$ On définit : $f=x\associe2-|3.x-2|$ et $g=x\associe[x]+f(x-[x])$.
Représentez $f$ et $g$. Montrez que $g$ est continue et lipschitzienne.\\
\begin{minipage}[b][1\totalheight][t]{0.62\columnwidth}%
Montrez que pour tout $\lambda$ réel l'équation $g(x)=\lambda$ d'inconnue
$x$ a exactement trois solutions.\\
Montrez qu'il n'existe pas d'application $h$ continue de $\R$ dans
$\R$ telle que pour tout réel $\lambda$ l'équation $h(x)=\lambda$
ait exactement deux solutions.%
\end{minipage}%
\begin{minipage}[b][1\totalheight][t]{0.37\columnwidth}%
~~\includegraphics[width=6cm,height=4cm,bb = 0 0 200 100, draft, type=eps]{imageTD18_5.png}%
\end{minipage}\\

\nex Petit quiz très formateur.\\
si $f$ est %
\begin{tabular}{|c|}
\hline 
\tabularnewline
\hline 
\end{tabular} sur $]-\infty,\ 0]$ et sur $[0,\ +\infty[$ alors elle l'est aussi
sur tout $\R$.\\
Quels mots pouvez mettre dans le cadre : %
\begin{tabular}{|c|c|c|}
\hline 
croissante & continue & monotone\tabularnewline
\hline 
\hline 
bornée & continue à droite & lipschitzienne\tabularnewline
\hline 
\end{tabular}\\
Donnez un argument, ou sinon un contre-exemple.\\

Même question avec \og si $f$ est %
\begin{tabular}{|c|}
\hline 
\tabularnewline
\hline 
\end{tabular} sur $]-\infty,\ 0]$ et sur $]0,\ +\infty[$ alors elle l'est aussi
sur tout $\R$\fg . (pas pareil...)\\

\nex $\heartsuit$ Une application est dite lilipschitzienne si il
existe $K$ vérifiant $\forall(x,y),\ |f(x)-f(y)|\pp k.|x-y|^{2}$.
Montrez que les applications lilipschitziennes ont une dérivée nulle.
Déduisez que l'ensemble des applications lilipschiztiennes de $\R$
dans $\R$ est un espace vectoriel de dimension $1$.\\

\nex $\heartsuit$ Soit $f$ continue de $\R$ dans $\R$ admettant
une limite en $+\infty$ et en $-\infty$. Montrez que $f$ est bornée.\\
La méthode astucieuse consistera à définition $\varphi=\theta\associe f(\tan(\theta))$
sur $]-\pi/2,\ \pi/2[$ et même si possible $]-\pi/2,\ \pi/2[$.\\

\nex $\spadesuit$ $n$ est un entier naturel ; résolvez l'équation
$z^{2.n}+1=0$ \emph{\small{}(en passant par $z=r.e^{i.\theta}$)}
puis factorisez $(X^{2.n}+1)$ dans $\C$.\\
On pose $f_{x}=\theta\associe\ln(x^{2}-2.x.\cos(\theta)+1)$ \emph{\small{}($x$
est un réel fixé de $]1,\ +\infty[$)}. Montrez que $f_{x}$ est intégrable
sur $[0,\ \pi]$.\\
Exprimez grâce à la première question la somme de Riemann milieu de
$f_{x}$ pour l'équisubdivision de $[0,\ \pi]$.\\
Déduisez : $\dsp\int_{0}^{\pi}\ln(x^{2}-2.x.\cos(\theta)+1).d\theta=2.\pi.\ln(x)$.\\
Calculez $\dsp\int_{0}^{\pi}\ln(x^{2}-2.x.\cos(\theta)+1).d\theta$
pour $x$ dans $]-\infty,\ -1[$ (indication : $t=\pi-\theta$).\\
Calculez $\dsp\int_{0}^{\pi}\ln(x^{2}-2.x.\cos(\theta)+1).d\theta$
pour $x$ dans $]-1,\ 1[$.\\
Vérifiez que $\dsp x\associe\int_{0}^{\pi}\ln(x^{2}-2.x.\cos(\theta)+1).d\theta=2.\pi.\ln(x)$
admet la même limite à droite et à gauche en $1$, dont on estimera
que c'est la valeur de $\dsp\int_{0}^{\pi}\ln(1^{2}-2.1.\cos(\theta)+1).d\theta$.
Calculez alors $\dsp\int_{0}^{\pi}\ln(\sin(t)).dt$.\\

\nex En quels point la dérivée de $x\associe|\ln(|x|)|$ est elle
discontinue ?\\

\nex Parmi ces trois applications, combien sont lipschitziennes sur
$\R$ \\
\begin{tabular}{|c|c|c|}
\hline 
$x\associe x^{2}$ & $x\associe Arctan(x)$ & $x\associe Arctan(x^{2})$\tabularnewline
\hline 
\hline 
\includegraphics[width=4.7cm,height=2.3cm,bb = 0 0 200 100, draft, type=eps]{imagemardi21_1.png} & \includegraphics[width=4.7cm,height=2.3cm,bb = 0 0 200 100, draft, type=eps]{imagemardi21_2.png} & \includegraphics[width=4.7cm,height=2.3cm,bb = 0 0 200 100, draft, type=eps]{imagemardi21_3.png}\tabularnewline
\hline 
\end{tabular}\\

\nex Qui, de $\theta\associe\sin(\tan(\theta))$ et $\theta\associe\tan(\sin(\theta))$
est lipschitzienne sur $]-\pi/2,\ \pi/2[$ ?\\

\nex Montrez que l'image d'une suite de Cauchy par une application
lipschitzienne est une suite de Cauchy.\\

\nex Montrez que $x\associe\cos(x)$ et $x\associe\cos(\sqrt{2}.x)$
sont périodiques. \\
On suppose que $x\associe\cos(x)+\cos(\sqrt{2}.x)$ est périodique
de période $p$. Montrez alors $\cos(p)+\cos(\sqrt{2}.p)=2$. Déduisez
$\cos(p)=1$ et $\cos(\sqrt{2}.p)=1$. Concluez.\\

\nex $\heartsuit$ Montrez que si $f$ est lipschitzienne de $]-\infty,\ 0]$
dans $\R$ (rapport $H$) et de $[0,\ +\infty[$ dans $\R$ (rapport
$K$) alors elle l'est de $\R$ dans $\R$.\\

\nex Montrez que si $f$ est lipschitzienne de $\R$ dans $\R$,
alors $|f|$ l'est aussi.\\
Montrez que $x\associe(-1)^{[x]}$ n'est pas lipschitzienne de $\R$
dans $\R$ mais que sa valeur absolue l'est aussi.\\

On suppose $|f|$ lipschitzienne de $\R$ dans $\R$ (rapport $K$),
et continue.\\
On se donne $a$ et $b$. Montrez que si $f(a)$ et $f(b)$ sont de
même signe, alors on a $|f(b)-f(a)|\leqslant K.|b-a|$.\\
On les suppose cette fois de signes opposés. Montrez qu'il existe
$c$ entre $a$ et $b$ vérifiant $f(c)=0$. Montrez alors \\
$|f(b)-f(a)|\leqslant|f(b)|+|f(a)|\leqslant K.|c-b|+K.|c-a|\leqslant K.|b-a|$.\\
Concluez : $f$ est à son tour lipschitzienne.\\

\nex Construire une suite qui admet une sous-suite strictement croissante
de limite $1$ et une sous-suite strictement décroissante de limite
$0$.\\

\nex $\heartsuit$ Montrez que $x\associe\tan(x)$ est lipschitzienne
de $[-\pi/3,\ \pi/3]$ dans $\R$ mais pas de $[0,\ \pi/2[$ dans
$\R$ (on pourra montrer que sinon, elle serait bornée).\\

\nex Montrez que $x\associe x.\sin(x)$ est lipschitzienne sur chaque
segment $[-a,\ a]$. Montrez qu'elle n'est pas lipschitzienne sur
$\R$.\\
\includegraphics[width=0.49\columnwidth,bb = 0 0 200 100, draft, type=eps]{imageTD20_4.png}
\includegraphics[width=0.49\columnwidth,bb = 0 0 200 100, draft, type=eps]{imageTD20_5.png}
\\
Et pour $x\associe x.\sin(\sqrt{x})$ ?\\

\nex Montrez que si $f$ et $[f]$ sont lipschitziennes de $\R$
dans $\R$, alors $f$ est bornée (raisonnez, n'écrivez pas plein
de formules).\\

\nex Montrez que $x\associe\cos(e^{x})$ n'est pas lipschitzienne
(étudiez le taux d'accroissement entre $\ln(k.\pi)$ et $\ln((k+1).\pi)$
pour $k$ entier).\\

\nex Montrez que l'application tangente est lipschitzienne sur $[-a,\ a]$
pour $a$ strictement plus petit que $\pi/2$ \emph{\small{}(sans
dériver, mais en pensant à $\dsp\frac{\sin(b-a)}{\cos(b).\cos(a)}$)}.\\
Lipschitzienne sur $I$ c'est $\exists K\in\R^{+},\ \forall(x,\ y)\in I^{2},\ |f(x)-f(y)|\leqslant K.|x-y|$.\\
Au fait, quelles applications vérifient $\exists K\in\R^{+},\ \forall(x,\ y)\in I^{2},\ |f(x)-f(y)|<K.|x-y|$
?\\

\nex $\heartsuit$ Montrez que $\ln\Big(\dsp\frac{1+2/k^{2}}{1+1/k^{2}}\Big)$
est équivalent à $\dsp\frac{1}{k^{2}}$ quand $k$ tend vers l'infini.
Déduisez que le produit $\dsp\prod_{k=0}^{n}\frac{k^{2}+1}{k^{2}+2}$
admet une limite quand $n$ tend vers l'infini. Écrivez un script
\texttt{\textbf{Python}} qui calcule la valeur approchée de ce produit
pour $n$ égal à $100$.\\
Qu'en est il du produit $\dsp\prod_{k=0}^{n}\frac{k+1}{k+2}$ ?\\

\nex Pour tout $n$, on note $a_{n}$ la solution dans $]0,\ +\infty[$
de l'équation $\ln(x)=n.\pi+Arctan(x)$ (existence ? unicité ?).\\
Montrez que la suite $(a_{n})$ est croissante et tend vers $+\infty$.\\
Montrez que la série de terme général $(a_{n})_{n\geqslant0}$ diverge.\\

\nex $\heartsuit$ Résolvez $z+\overline{z}=|z|$ d'inconnue $z$
dans $\C$.{\tiny{}Paques05}\\

\nex La suite $(u_{n})$ est définie par $\sqrt{n+1}-\sqrt{n}=\dsp\frac{1}{2.\sqrt{n+u_{n}}}$.
Donnez sa limite quand $n$ tend vers l'infini.\\

\nex $A$ et $B$ sont deux sous espaces vectoriels de $(E,+,.)$
(espace de dimension finie) vérifiant $A+B=E$.\\
Montrez qu'il existe %
\begin{tabular}{cccc}
$C$ & sous espace vectoriel de $A$ & vérifiant & $C\oplus B=E$\tabularnewline
$D$ & sous espace vectoriel de $D$ & vérifiant & $A\oplus D=E$\tabularnewline
\end{tabular}.\\
Que pouvez vous dire si $C+D=E$ ?\\

\nex $f$ et $g$ sont deux applications continues de $\R^{+}$ dans
$\R$. On suppose $f(x)\sim_{x\rightarrow+\infty}g(x)$ \emph{\small{}(au
fait, pour un demi point déjà, ça veut dire quoi ?)}. Lesquelles de
ces affirmations sont alors vraies : \\
\begin{tabular}{|c|c|c|c|c|}
\hline 
A & si & $f$ est bornée & alors & $g$ est bornée\tabularnewline
\hline 
B & si & $f$ est dérivable en tout point & alors & $g$ est dérivable en tout point\tabularnewline
\hline 
C & si & $f$ est positive en tout point & alors & $g$ est positive en tout point\tabularnewline
\hline 
D & si & $f$ est périodique & alors & $g$ est périodique à partir d'un certain réel\tabularnewline
\hline 
E & si & $f$ est croissante & alors & $g$ est croissante\tabularnewline
\hline 
F & si & $f$ est lipschitzienne & alors & $g$ est lipschitzienne\tabularnewline
\hline 
\end{tabular}{\tiny{}Compil2018}\\

\nex $\heartsuit$ Montrez que la probabilité que \texttt{\textbf{randrange(1,
626)}} soit un carré parfait est de $\dsp\frac{1}{25}$.\\
Quelle est la probabilité que \texttt{\textbf{randrange(1, 626){*}{*}2}}
soit un carré parfait ?\\
Quelle est la probabilité que \texttt{\textbf{randrange(1, 626){*}randrange(1,
626)}} soit un carré parfait ? (exploitez Python si nécessaire pour
dénombrer).\\

\trait\nqp On note $E$ l'ensemble des applications lipschitziennes
de $[0,\ 1]$ dans $\R$. Montrez que $(E,+,.)$ est un espace vectoriel.\\
\nq Montrez que $f\associe||f||$ est une norme sur m(E,+,.) , sachant
que l'on pose $||f||=Sup(|f(t)|\ |\ t\in[0,\ 1])$.\\
\begin{minipage}[t]{0.98\columnwidth}%
N est une norme sur $(F,+,.)$ où $F$ est un espace vectoriel \\
\begin{minipage}[b][1\totalheight][t]{0.26\columnwidth}%
{\scriptsize{}Le mnémotechnique pour retenir cette liste, c'est ``Sophie
a perdu son haut'' si vous reprenez l'idée de François-Xavier il
y a déjà dix huit ans de ça, et si vous voulez j'ai des photos de
la Sophie en question.}%
\end{minipage} %
\noindent\begin{minipage}[b][1\totalheight][t]{0.71\columnwidth}%
\begin{tabular}{|c|c|c|}
\hline 
E & {\small{}Existence} & {\small{}pour tout $\overrightarrow{u}$ de $E$, $N(\overrightarrow{u})$
existe}\tabularnewline
\hline 
P & {\small{}Positivité} & {\small{}$\forall\overrightarrow{u}\in E,\ N(\overrightarrow{u})\geqslant0$}\tabularnewline
\hline 
S & {\small{}Séparation} & {\small{}}%
\begin{minipage}[t]{0.5\columnwidth}%
{\small{}$\forall\overrightarrow{u}\in E,\ \overrightarrow{u}\neq\overrightarrow{0}\Rightarrow N(\overrightarrow{u})>0$
}\\
{\small{}$\forall\overrightarrow{u}\in E,\ N(\overrightarrow{u})=0\Rightarrow\overrightarrow{u}=\overrightarrow{0}$}%
\end{minipage}\tabularnewline
\hline 
H & {\small{}Homogénéité} & {\small{}$\forall(\lambda,\ \overrightarrow{u})\in\R\times E,\ N(\lambda.\overrightarrow{u})=|\lambda|.N(\overrightarrow{u})$}\tabularnewline
\hline 
I & {\small{}Inégalité triangulaire} & {\small{}$\forall(\overrightarrow{u},\ \overrightarrow{v})\in E^{2},\ N(\overrightarrow{u}+\overrightarrow{v})\leqslant N(\overrightarrow{u})+N(\overrightarrow{v})$}\tabularnewline
\hline 
\end{tabular}%
\end{minipage}%
\end{minipage}\\
\begin{minipage}[t]{0.98\columnwidth}%
{\small{}Je vous donne quand même le début d'une des preuves pour
que vous ne vous contentiez pas d'affirmations péremptoires ``il
est évident que'' :}\\
{\small{}on se donne $f$ et $g$ ; pour tout $x$, on a : $|f(x)+g(x)|\leqslant|f(x)|+|g(x)|\leqslant||f||+||g||$
et ensuite, à vous de rédiger avec des mots et pas avec des trucs
dont vous dites que c'est des maths...}%
\end{minipage}\\
\nq Pour $f$ dans $E$, on pose $L(f)=Sup\Big\{\dsp\frac{|f(b)-f(a)|}{|b-a|}\ |\ 0\leqslant a<b\leqslant1\Big\}$.
Montrez que $L$ est une semi norme sur $(E,+,.)$ \emph{\small{}(semi-norme,
c'est EPHI)}.\\
\nq Montrez pour $f$ de classe $C^{1}$ : $L(f)=||f'||$. \emph{\small{}Et
si vous préférez des photos de François-Xavier, j'ai aussi.}\\
\nq Calculez la norme $||f||$ et la semi-norme pour les applications
suivantes : \\
\begin{tabular*}{1\columnwidth}{@{\extracolsep{\fill}}|c|c|c|c|c|}
\hline 
$s_{n}=\theta\associe\sin(n.\theta)$ & $c_{n}=\theta\associe\cos(n.\theta)$ & $x\associe x^{n}$ & $x\associe|2.x-1|$ & $x\associe\ln(1+x)$\tabularnewline
\hline 
\end{tabular*}\\
\nq Montrez que $f\associe|f(0)|+L(f)$ est une norme \emph{\small{}(notée
$\Lambda$)}. Montrez pour tout $f$ de $||f||\leqslant\Lambda(f)$.
Existe-t-il $K$ vérifiant $\forall f\in E,\ L(f)\leqslant K.||f||$
.\\

\trait\nqp Une suite $(f_{n})$ d'éléments de $E$ vérifie $\forall\varepsilon,\ \exists K_{\varepsilon},\ \forall(p,q),\ K_{\varepsilon\leqslant}p\leqslant q\Rightarrow L(f_{p}-f_{q})\leqslant\varepsilon$.
Montrez que pour tout $x$ de $[0,\ 1]$, la suite $(f_{n}(x))$ converge
vers un réel que l'on va noter $f(x)$.\\
\nq Montrez que $f$ ainsi définie (limite des $f_{n}$) est dans
$E$.\\

\trait\nqp Pour tout $x$ de $[0,\ 1]$ et tout $n$ on pose $F_{n+1}(x)=F_{n}(x)+\dsp\frac{x-(F_{n}(x))^{2}}{2}$
et $F_{0}(x)=0$. Montrez que chaque $F_{n}$ est un polynôme et donnez
son degré. Chaque $P_{n}$ est il dans $E$ ?\\
\nq Montrez que la suite $(F_{n}(x))$ est croissante majorée et
converge (étudiez $t\associe\dsp t+\frac{x-t^{2}}{2}$ sur $[0,\ 1]$).\\
\nq La limite des $P_{n}$ est elle dans $E$ ?\\

\end{document}
